% Stolen from: https://stackoverflow.com/questions/1603301/how-to-add-page-numbers-to-postscript-pdf
\documentclass[8pt]{report}
\usepackage[final]{pdfpages}

%\topmargin 70pt
%\oddsidemargin 150pt
%\evensidemargin -40pt

\title{Proceedings of the\\Functional Programming Lab\\Away Day 2014}

\author{\Huge$\lambda$}

\date{3-4 July, 2014\\Old Hall Hotel, Buxton\\\vfill The University of Nottingham}

\begin{document}

\maketitle

\tableofcontents

% page number - 1, section, level, heading, label
\includepdfset{pages=-,pagecommand=\thispagestyle{plain},addtotoc={
  1,chapter,0, Extended Abstract: Bridging the GUI Gap with Reactive Values and Relations ,p1,
  2,chapter,0, An Investigation Into the Use of Haskell for Dynamic Programming           ,p2,
  3,chapter,0, Pick'n'Fix: Modular Control Structures                                     ,p3,
  4,chapter,0, Worker/Wrapper/Makes it/Faster                                             ,p4,
  5,chapter,0, Calculating Correct Compilers                                              ,p5,
  6,chapter,0, Attribute Grammars and Containers                                          ,p6,
  7,chapter,0, A principled approach to the implementation of argumentation models        ,p7,
  8,chapter,0, Definability and Kripke Logical Relations                                  ,p8,
  9,chapter,0, A syntax for cubical type theory                                           ,p9,
  10,chapter,0, Constructing 2-HITs                                                        ,p10,
  12,chapter,0, Mutual and Higher Inductive Types in Homotopy Type Theory                  ,p11,
  13,chapter,0, Omega-Constancy and Truncations                                            ,p12,
  14,chapter,0, Definable quotients in intensional type theory                             ,p13
}}
\includepdf{papers.pdf}


\end{document}
